\documentclass[]{article}
\usepackage{lmodern}
\usepackage{amssymb,amsmath}
\usepackage{ifxetex,ifluatex}
\usepackage{fixltx2e} % provides \textsubscript
\ifnum 0\ifxetex 1\fi\ifluatex 1\fi=0 % if pdftex
  \usepackage[T1]{fontenc}
  \usepackage[utf8]{inputenc}
\else % if luatex or xelatex
  \ifxetex
    \usepackage{mathspec}
  \else
    \usepackage{fontspec}
  \fi
  \defaultfontfeatures{Ligatures=TeX,Scale=MatchLowercase}
\fi
% use upquote if available, for straight quotes in verbatim environments
\IfFileExists{upquote.sty}{\usepackage{upquote}}{}
% use microtype if available
\IfFileExists{microtype.sty}{%
\usepackage{microtype}
\UseMicrotypeSet[protrusion]{basicmath} % disable protrusion for tt fonts
}{}
\usepackage[margin=1in]{geometry}
\usepackage{hyperref}
\hypersetup{unicode=true,
            pdftitle={Estimadores de Máxima Verossimilhança para Assimetria e Curtose usando a expansão de Gram-Charlier},
            pdfauthor={Wilson Freitas},
            pdfborder={0 0 0},
            breaklinks=true}
\urlstyle{same}  % don't use monospace font for urls
\usepackage{longtable,booktabs}
\usepackage{graphicx,grffile}
\makeatletter
\def\maxwidth{\ifdim\Gin@nat@width>\linewidth\linewidth\else\Gin@nat@width\fi}
\def\maxheight{\ifdim\Gin@nat@height>\textheight\textheight\else\Gin@nat@height\fi}
\makeatother
% Scale images if necessary, so that they will not overflow the page
% margins by default, and it is still possible to overwrite the defaults
% using explicit options in \includegraphics[width, height, ...]{}
\setkeys{Gin}{width=\maxwidth,height=\maxheight,keepaspectratio}
\IfFileExists{parskip.sty}{%
\usepackage{parskip}
}{% else
\setlength{\parindent}{0pt}
\setlength{\parskip}{6pt plus 2pt minus 1pt}
}
\setlength{\emergencystretch}{3em}  % prevent overfull lines
\providecommand{\tightlist}{%
  \setlength{\itemsep}{0pt}\setlength{\parskip}{0pt}}
\setcounter{secnumdepth}{0}
% Redefines (sub)paragraphs to behave more like sections
\ifx\paragraph\undefined\else
\let\oldparagraph\paragraph
\renewcommand{\paragraph}[1]{\oldparagraph{#1}\mbox{}}
\fi
\ifx\subparagraph\undefined\else
\let\oldsubparagraph\subparagraph
\renewcommand{\subparagraph}[1]{\oldsubparagraph{#1}\mbox{}}
\fi

%%% Use protect on footnotes to avoid problems with footnotes in titles
\let\rmarkdownfootnote\footnote%
\def\footnote{\protect\rmarkdownfootnote}

%%% Change title format to be more compact
\usepackage{titling}

% Create subtitle command for use in maketitle
\newcommand{\subtitle}[1]{
  \posttitle{
    \begin{center}\large#1\end{center}
    }
}

\setlength{\droptitle}{-2em}
  \title{Estimadores de Máxima Verossimilhança para Assimetria e Curtose usando a
expansão de Gram-Charlier}
  \pretitle{\vspace{\droptitle}\centering\huge}
  \posttitle{\par}
  \author{Wilson Freitas}
  \preauthor{\centering\large\emph}
  \postauthor{\par}
  \predate{\centering\large\emph}
  \postdate{\par}
  \date{18 de abril de 2018}

\usepackage{booktabs}
\usepackage{longtable}
\usepackage{array}
\usepackage{multirow}
\usepackage[table]{xcolor}
\usepackage{wrapfig}
\usepackage{float}
\usepackage{colortbl}
\usepackage{pdflscape}
\usepackage{tabu}
\usepackage{threeparttable}
\usepackage[normalem]{ulem}

\begin{document}
\maketitle

\section{Resumo}\label{resumo}

A estimação da assimetria e da curtose em séries de retornos é explorada
neste artigo com o objetivo de prover uma alternativa aos estimadores
amostrais comumente utilizados. Os estimadores amostrais para assimetria
e curtose apresentam grande sensibilidade a novas ocorrências nas séries
temporais. Esta sensibilidade é inerente a estes estimadores, sendo
difícil de ser explicada pela dinâmica observada nos dados. O objetivo
deste texto é apresentar estimadores de assimetria e curtose menos
sensíveis, que consideram uma expansão em série de momentos, a expansão
Gram-Charlier. A expansão de Gram-Charlier é aplicada a distribuição
Gaussiana na construção de uma função de densidade de probabilidade
paramétrica com relação a assimetria e curtose. Utilizamos o método de
máxima verossimilhança (MV) para obter a assimetria e curtose na
expansão de Gram-Charlier. Observamos mais estabilidade nestes
estimadores do que nos amostrais. Esta estabilidade é importante na
aplicação da expansão de Gram-Charlier a problemas de finanças, por
exemplo o apreçamento de opções.

\section{Introdução}\label{introducao}

Qualquer função de densidade de probabilidade (FDP) pode ser aproximada
por uma expansão em série de outra FDP que pertence a um grupo especial,
como a distribuição Gaussiana, por exemplo. Dada uma amostra, é difícil
identificar a melhor distribuição para descrevê-la, logo, a utilização
expansões em séries permite associar uma FDP aproximada aos dados. A FDP
aproximada é uma expansão em série de uma distribuição conhecida e em
geral padronizada, esta é a distribuição base. Esta aproximação permite
lidar com momentos que não estão presentes na distribuição base. Por
exemplo, a distribuição Gaussiana não possui os momentos de assimetria e
curtose, entretanto, na expansão em série da Gaussiana estes momentos
são parâmetros. Portanto, se os dados apresentam assimetria e curtose, a
distribuição Gaussiana não é uma candidata, contudo, a sua expansão é.
As expansões de FDPs em séries têm aplicações em diversos campos da
ciência como a Física e Finanças. Este texto explora o uso da expansão
de Gram-Charlier ~{[}1{]} que é a expansão da distribuição Gaussiana em
uma base de polinônmios de Hermite.

\section{Assimetria e curtose}\label{assimetria-e-curtose}

Seja uma amostra aleatória \(\{x_i\}_{i=1...N}\), de média amostral
\(\mu\) e variância amostral \(\sigma^2\). Os estimadores amostrais de
assimetria \(\mu_3\) e curtose \(\mu_4\) são definidos como:

\[
\begin{split}
\mu_3 & = \frac{1}{N} \sum_{i=1}^{N} \frac{(x_i - \mu)^3}{\sigma^3} \\ \\
\mu_4 & = \frac{1}{N} \sum_{i=1}^{N} \frac{(x_i - \mu)^4}{\sigma^4}
\end{split}
\]

Estes estimadores apresentam grande sensibilidade a pequenas mudanças na
amostra. Em séries de retornos diários observam-se mudanças de nível
devido a presença retornos atípicos causados por fortes oscilações no
mercado. Há casos onde apenas um retorno atípico causa uma variação
significativa. Estas fortes mudanças podem ser notadas ao se calcular a
assimetria e curtose em uma janela móvel aplicada a série de retorno.
Observam-se fortes mudanças de regime em períodos de aumento
volatilidade no mercado. A Figura \ref{fig:skewness-kurtosis-ma}
apresenta os gráficos da série de retornos padronizados do IBOVESPA e os
estimadores de assimetria, curtose e do desvio padrão em janela móvel. A
série de retornos do IBOVESPA é referente ao período de janeiro de 2000
a dezembro de 2017 (17 anos) e os retornos são contínuos. A janela móvel
de dois anos (504 dias úteis) é utilizada para o cálculo das médias
móveis.

\begin{figure}
\centering
\includegraphics{estimadores-assimetria-curtose_files/figure-latex/unnamed-chunk-3-1.pdf}
\caption{\label{fig:skewness-kurtosis-ma} Os gráficos de retornos
padronizados do IBOVESPA (i) e suas médias móveis de dois anos: (ii)
desvio padrão, (iii) curtose e (iv) assimetria.}
\end{figure}

Nos gráficos de média móvel de assimetria e curtose, na Figura
\ref{fig:skewness-kurtosis-ma}, é possível observar que em diversos
momentos há mudanças de nível significativas, relacionadas a períodos de
maior volatilidade nos retornos. Entretanto, o gráfico com a média móvel
do desvio padrão, apresenta um comportamento mais suave, nos mesmos
períodos.

A expansão de Gram-Charlier é apresentada na próxima sessão. Através
desta expansão chega-se a uma distribuição paramétrica na assimetria e
na curtose. A partir desta distribuição é possível obter estimadores de
máxima verossimilhança para estas grandezas relacionadas a esta
distribuição.

\section{Expansão de Gram-Charlier}\label{expansao-de-gram-charlier}

Sob certas condições uma função de distribuição de probabilidade (FDP)
\(p(x)\), padronizada, pode ser expandida em uma série de derivadas da
distribuição Gaussiana padronizada
\(Z(x) = \frac{1}{\sqrt{2\pi}}\exp\left(-\frac{x^2}{2}\right)\). A
expansão em série

\[
p(x) \sim \sum_{n=0}^\infty c_n \frac{d^n Z(x)}{dx^n}
\]

é uma série de Gram-Charlier (de tipo-A) e de acordo com a fórmula de
Rodriguez

\[
He_n(x) = (-1)^n \exp\left( \frac{x^2}{2} \right) \frac{d^n}{dx^n} \exp\left( -\frac{x^2}{2} \right)
\]

são polinômios de Chebyshev-Hermite de ordem \(n\). Substituindo a
fórmula de Rodriguez na equação de \(p(x)\) obtem-se uma expansão em
série de polinômios de Chebyshev-Hermite e da distribuição Gaussiana.

\[
p(x) \sim \sum_{n=0}^\infty c_n (-1)^n He_n(x) Z(x)
\]

onde os coeficientes

\[
c_n = \frac{(-1)^n}{n!} \int_{-\infty}^{\infty} p(t) He_n(t) dt
\]

são valores esperados dos polinômios de Chebyshev-Hermite com relação a
FDP \(p(x)\). Os coeficientes \(c_n\) são funções dos momentos da
distribuição \(p(x)\), logo, a expansão de Gram-Charlier é uma expansão
em momentos da distribuição \(p(x)\).

Assumindo uma abordagem parsimoniosa e a dificuldade em se trabalhar com
séries infinitas, é razoável truncar a expansão \(p(x)\) em uma
quantidade finita de termos, \(n < \infty\). Uma boa escolha é \(n=4\),
pois, a expansão vai até os polinômios de ordem 4 gerando assim os
momentos de assimetria e curtose. Esta escolha produz a seguinte
aproximação para \(p(x)\)

\[
p(x) \sim Z(x)\left[ 1 + \frac{\mu_3}{3!} (x^3 - 3x) + \frac{\mu_4 - 3}{4!} (x^4 - 6x^2 + 3) \right]
\]

Note que além da distribuição Gaussiana \(Z(x)\), \(p(x)\) é função dos
terceiro e quarto momentos, \(\mu_3 = E[x^3]\) e \(\mu_4 = E[x^4]\) da
variável aleatória \(x\) padronizada. Logo, estes momentos centralizados
são assimetria e curtose, respectivamente. O truncamento da expansão de
Gram-Charlier em termos até a ordem \(n=4\) produz uma função
paramétrica na assimetria e na curtose, logo,
\(p(x) \equiv p(x;\mu_3,\mu_4)\). Esta função é uma extensão da
distribuição Gaussiana que incorpora assimetria e curtose, entretanto,
para que \(p(x)\) seja considerada uma FDP é necessário que as seguintes
restrições sejam atendidas:

\begin{itemize}
\tightlist
\item
  \(\int_{-\infty}^{\infty}p(x) dx = 1\)
\item
  \(p(x)\) seja contínua em todo suporte
\item
  \(p(x)\) seja não negativa, para todo \(x\)
\end{itemize}

Os dois primeiros pontos são fáceis de comprovar, integrando e derivando
\(p(x)\). Contudo, o terceiro ponto não é direto. Há de fato, valores de
assimetria e curtose para os quais a função \(p(x)\) é negativa. Na
próxima sessão serão apresentadas as retrições de positividade para
\(p(x)\), nas quais os momentos podem assumir valores onde esta função
seja sempre não negativa. Uma vez atendida a restrição de positividade,
podemos considerar \(p(x)\) uma FDP e ela poderá ser utilizada para
descrever variáveis aleatórias como séries de retornos. Estas séries
comumente apresentam desvios da normalidade, como retornos extremos e
oscilações assimétricas.

\subsection{\texorpdfstring{Restrições de positividade para
\(p(x)\)}{Restrições de positividade para p(x)}}\label{restricoes-de-positividade-para-px}

Seja a função \(p(x;\mu_3, \mu_4)\) definida na sessão anterior

\[
p(x; \mu_3, \mu_4) = Z(x)\left[ 1 + \frac{\mu_3}{3!} (x^3 - 3x) + \frac{\mu_4 - 3}{4!} (x^4 - 6x^2 + 3) \right]
\]

Esta função apresenta valores negativos de \(p(x)\) para alguns valores
de assimetria e curtose. Isso pode ser observado escrevendo esta função
com uma função de log-verossimilhança \(l(\mu_3, \mu_4; \mathbf{x})\),
que é função dos momentos e é paramétrica em
\(\mathbf{x} \equiv \{x_i\}_{i=1...N}\), uma amostra aleatória. Logo, a
função

\[
l(\mu_3, \mu_4; \mathbf{x}) = \sum_{i=1}^N - \log p(x_i; \mu_3, \mu_4)
\]

seria uma função \emph{pseudo} log-verossimilhança, onde \emph{pseudo}
quer dizer que esta função diverge para alguns valores dos momentos.
Isso acontece porque a função \(\log\) possui apenas suporte positivo e
valores negativos de \(p(x)\) apresentarão descontinuidades quando
calculados numericamente.

Para observar a região de \emph{validade} dos momentos na função
\(p(x)\) será gerado um \emph{grid} com possíveis valores de \(\mu_3\) e
\(\mu_4\) e para cada par será calculada a função
\(l(\mu_3, \mu_4; \mathbf{x})\) para uma amostra aleatória. O resultado
será visualizado em um gráfico 2D onde o eixo y é a assimetria e o eixo
x a curtose. Os pontos no gráfico são os valores obtidos da função
\(l(\mu_3, \mu_4; \mathbf{x})\) referente ao par de assimetria e
curtose. Os valores da função \(l(\mu_3, \mu_4; \mathbf{x})\) serão
segmentados de maneira que possamos observar a região do espaço
paramétrico onde a função apresenta o seu máximo. Nesta simulação
considera-se uma amostra aleatória normal com 1000 elementos e os
momentos gerados de acordo com os seguintes intervalos:
\(\mu_3 \in [-1.2, 1.2]\) e \(\mu_4 \in [2, 7]\).

\begin{figure}
\centering
\includegraphics{estimadores-assimetria-curtose_files/figure-latex/unnamed-chunk-6-1.pdf}
\caption{\label{fig:random-heatmap} O gráfico acima é um
\textit{heatmap} da função \(l(\mu_3, \mu_4; \mathbf{x})\) para uma
amostra aleatória normal. Os valores da função de log-verossimilhança
são segmentados de forma que é possível identificar que a região em
torno dos momentos populacionais, \(\mu_3=0\) e \(\mu_4 = 3\), apresenta
a maior verossimilhança. O ponto marcado com X indica os valores
amostrais de assimetria e curtose. Os pontos escuros onde a função de
log-verissimilhança é NA indicam que houve divergência no cálculo.
Portanto, na região definida por estes pontos a função \(p(x)\)
apresenta valores negativos, assim como na região de pontos coloridos
\(p(x)\) apresenta valores não negativos.}
\end{figure}

Observando a Figura \ref{fig:random-heatmap} fica claro que há uma
região bem definida para a qual os valores de assimetria e curtose
produzem uma função \(p(x)\) não negativa e portanto, nesta região,
\(p(x)\) pode ser considerada uma FDP.

A título de ilustração a Figura \ref{fig:ibovespa-heatmap} mostra o
\emph{heatmap} onde a série de retornos padronizados do IBOVESPA é
utilizada como amostra aleatória.

\begin{figure}
\centering
\includegraphics{estimadores-assimetria-curtose_files/figure-latex/unnamed-chunk-9-1.pdf}
\caption{\label{fig:ibovespa-heatmap} O gráfico mostra o
extit\{heatmap\} da função de log-verossimilhança aplicada a série de
retornos padrinizados do IBOVESPA no período de jan/2000 a dez/2017.
Nota-se que a região de máxima verossimilhança desloca-se para onde a
assimetria é negativa e a curtose é um pouco acima de 4, indicando uma
distribuição assimétrica e com caudas pesadas. Entretanto, os
estimadores amostrais ficam próximos a borda da região válida, muito
distantes da região de máxima verossimilhança.}
\end{figure}

Para utilizarmos a função de log-verossimilhança em um processo numérico
para estimar a assimetria e a curtose, é necessário encontrar uma forma
funcional para \(p(x)\) na qual seja possível tratar as restrições de
\(\mu_3\) e \(\mu_4\) de forma analítica e de preferência, sem
condicionais. Como observa-se nas Figuras \ref{fig:random-heatmap} e
\ref{fig:ibovespa-heatmap} há um plano de coordenadas \((\mu_3, \mu_4)\)
onde a região definida por uma figura oval nos gráficos é a região de
validade para os parâmetros \(\mu_3\) e \(\mu_4\). Vamos começar
determinando a borda dessa região.

Queremos valores para \(\mu_3\) e \(\mu_4\) tais que \(p(x)\) seja
positivo definido para qualquer \(x\). Observando \(p(x)\) temos que o
seguinte polinômio deve ser sempre positivo para qualquer \(x\).

\[
f(x) = 1 + \frac{\mu_3}{6} He_3(x) + \frac{(\mu_4 - 3)}{24} He_4(x) \geq 0
\] onde \(He_3(x) = x^3 - 3x\) e \(He_4(x) = x^4 - 6x^2 + 3\) são
polinômios de Chebyshev-Hermite. A borda dessa região é definida por

\[
f(x) = 1 + \frac{\mu_3}{6} He_3(x) + \frac{(\mu_4 - 3)}{24} He_4(x) = 0
\]

que define uma linha no plano \((\mu_3, \mu_4)\) para cada valor de
\(x\). Derivando \(f(x)\) e igualando a zero temos uma função que é
independente de \(x\), no sentido que para qualquer valor de \(x\) essa
derivada é nula.

\[
f^\prime(x) = \frac{\mu_3}{2} He_2(x) + \frac{(\mu_4 - 3)}{6} He_3(x) = 0
\]

Com isso podemos construir um sistema linear para encontrar \(\mu_3\) e
\(\mu_4\) como funções de \(x\), pois na primeira equação definimos a
restrição em relação a \(x\) e na segunda temos uma restrição mais geral
para qualquer valor de \(x\). As soluções \(\mu_3\) e \(\mu_4\) para
este sistema determinam a borda da região oval observada nas Figuras
\ref{fig:random-heatmap} e \ref{fig:ibovespa-heatmap}. A solução do
sistema é dada por:

\[
\begin{split}
\mu_3(x) &= -24\frac{He_3(x)}{d(x)} \\
\mu_4(x) &= 72\frac{He_2(x)}{d(x)} + 3
\end{split}
\]

onde \(d(x) = 4He_3^2(x) - 3He_2(x) He_4(x)\).

Agora é necessário analisar estas equações para determinar as regiões em
\(x\) para que \(\mu_3(x)\) e \(\mu_4(x)\) tenham valores válidos. A
Tabela \ref{tab:1} apresenta as regiões de \(x\) que atendem as
restrições para as equações de \(\mu_3(x)\) e \(\mu_4(x)\), sabendo que
\(\mu_3(x) \geq 0\) ou \(\mu_3(x) < 0\) e que \(\mu_4(x) \geq 3\).

\begin{longtable}[]{@{}ll@{}}
\caption{\label{tab:1} Regiões de \(x\) que atendem as restrições de
\(\mu_3(x)\) e \(\mu_4(x)\).}\tabularnewline
\toprule
Restrições & Regiões de \(x\)\tabularnewline
\midrule
\endfirsthead
\toprule
Restrições & Regiões de \(x\)\tabularnewline
\midrule
\endhead
\(\mu_3 < 0\) & \(x > \sqrt{3} \cup -\sqrt{3} < x < 0\)\tabularnewline
\(\mu_3 \geq 0\) &
\(x \leq -\sqrt{3} \cup 0 < x < \sqrt{3}\)\tabularnewline
\(\mu_4 < 3\) & \(-1 < x < 1\)\tabularnewline
\(\mu_4 \geq 3\) & \(x \geq 1 \cup x \leq -1\)\tabularnewline
\bottomrule
\end{longtable}

Tomando a intercecção das restrições válidas chegamos a 2 conjuntos para
\(x\)

\begin{itemize}
\tightlist
\item
  \(x \in (-\infty, -\sqrt{3}) \cup (\sqrt{3}, \infty)\) (menor)
\item
  \(x \in (-\sqrt{3}, \sqrt{3})\) (maior --- admite valores inválidos
  para \(\mu_4\))
\end{itemize}

Note que estes conjuntos definem regiões de \(x\) onde o polinômio
\(f(x)\) da expansão de Gram-Charlier é igual a zero e consequentemente,
os valores de \(\mu_3\) e \(\mu_4\) são válidos dentro dessa restrição.

\begin{figure}
\centering
\includegraphics{estimadores-assimetria-curtose_files/figure-latex/unnamed-chunk-10-1.pdf}
\caption{\label{fig:eggs} Regiões de assimetria e curtose para que a
expansão de Gram-Charlier seja não negativa. A região maior (azul), onde
\(x \in (-\sqrt{3}, \sqrt{3})\), admite valores negativos para curtose,
o que não é permitido, pois este momento é positivo. A menor região
(vermelho) admite valores válidos para assimetria e curtose e
\(x \in (-\infty, -\sqrt{3}) \cup (\sqrt{3}, \infty)\).}
\end{figure}

Como observa-se na Figura \ref{fig:eggs}, a busca de parâmetros
\((\mu_3, \mu_4)\) deve ser restrita a região oval em que
\(\mu_4 \in [3,7]\) e \(\mu_3 \equiv \mu_3(\mu_4)\). Na borda dessa
região tem-se que
\(x \in (-\infty, -\sqrt{3}) \cup (\sqrt{3}, \infty)\). Estas restrições
colocadas para \(\mu_3\) e \(\mu_4\) não são desejáveis para a
utilização de \(p(x)\) como uma FDP em um processo de estimação por
máxima-verossimilhança. Para realizar a estimação destes parâmetros por
máxima-verossimilhança seria necessário introduzir uma restrição
funcional aos parâmetros. Para contornar este problema vamos aplicar uma
transformação a \(\mu_3\) e \(\mu_4\) de maneira que seja possível
eliminar essa restrição criando novas variáveis \(\mu_3^\prime\) e
\(\mu_4^\prime\) irrestritas.

Considerando \(\mu_4 = g(\mu_4^\prime, \mu_{4,inf}, \mu_{4,sup})\) onde
\(g(x, l_{inf}, l_{sup}) = l_{inf} + \frac{(l_{sup} - l_{inf})}{1 - e^{-x}}\).
Dessa maneira, a partir de \(\mu_4^\prime\) (irrestrito) encontramos
\(\mu_4\) e com isso aplicamos a função que delimita a região oval para
encontrar os limites de \(\mu_3\) (\(\mu_{3,inf}\) e \(\mu_{3,sup}\)
como função de \(\mu_4\)). Utilizamos a função
\(\mu_3 = g(\mu_3^\prime, \mu_{3,inf}, \mu_{3,sup})\) para a partir de
\(\mu_3^\prime\) encontrar \(\mu_3\). Com essa transformação analítica
elimina-se o problema de restrição da busca por \(\mu_3\) e \(\mu_4\)
pois o problema passa a ser posto em termos de \(\mu_3^\prime\) e
\(\mu_4^\prime\). A Figura \ref{fig:limit-transform} mostra o gráfico da
função \(g(x)\) aplicada a curtose
\(\mu_4 = g(\mu_4^\prime, \mu_{4,inf}=3, \mu_{4,sup}=7)\).

Desta forma, a função de log-verossimilhança pode ser reescrita como
\(l(\mu_3(\mu_3^\prime), \mu_4(\mu_4^\prime); \mathbf{x}) \equiv l(\mu_3^\prime, \mu_4^\prime; \mathbf{x})\).
A função \(p(x)\) gerada a partir da expansão de Gram-Charlier passa a
atender a restrição de positividade e portanto pode ser considerada uma
FDP. Esta função pode ser utilizada para encontrar os estimadores de
máxima-verossimilhança para assimetria e curtose em um processo de
otimização sem restrições funcionais. Na próxima sessão será realizada a
estimação dos parâmetros para uma amostra aleatória normal e para a
série de retornos padronizados do IBOVESPA. Será possível notar as
diferenças entre esta abordagem e os estimadores amostrais.

\begin{figure}
\centering
\includegraphics{estimadores-assimetria-curtose_files/figure-latex/unnamed-chunk-11-1.pdf}
\caption{\label{fig:limit-transform} Função \(g(x)\) para realizar uma
transformação analítica em \(\mu_3\) e \(\mu_4\) de maneira tornar um
problema de busca restrito em um problema irrestrito. Aqui tem-se
\(g(\mu_4^\prime, 3, 7)\).}
\end{figure}

\section{Estimação da assimetria e curtose com a expansão de
Gram-Charlier}\label{estimacao-da-assimetria-e-curtose-com-a-expansao-de-gram-charlier}

Para realizar a estimação de assimetria e curtose via máximização da
função de verossimilhança, é necessário ter uma função de
verossimilhança e, para definí-la, precisamos de uma FDP que descreva a
amostra aleatória de interesse. Na sessão anterior foi demostrado que,
para uma região do espaço paramétrico de assimetria e curtose, a
expansão de Gram-Charlier, \(p(x)\), é atende às condições para que seja
caracterizada como uma FDP. A limitação do espaço paramétrico torna o
processo de busca pelos parâmetros restrito. Para contornar esta
restrição, foi realizada a transformação de variáveis de
\((\mu_3, \mu_4) \rightarrow (\mu_3^\prime, \mu_4^\prime)\), tornando o
problema restrito em \((\mu_3, \mu_4)\) em um problema irrestrito em
\((\mu_3^\prime, \mu_4^\prime)\). Foi definida a função de
log-verossimilhança \(l(\mu_3^\prime, \mu_4^\prime; \mathbf{x})\)
irrestrita, onde \(\mu_3 \equiv \mu_3(\mu_3^\prime)\) e
\(\mu_4 \equiv \mu_4(\mu_4^\prime)\). O processo de otimização para
encontrar a máxima verossimilhança pode ser realizado sem maiores
problemas no espaço \((\mu_3^\prime, \mu_4^\prime)\). Os valores ótimos
de \(\mu_3^\prime\) e \(\mu_4^\prime\) são encontrados e a partir destes
são obtidos os valores de assimetria e curtose.

A Tabela \ref{tab:mv-results} apresenta uma comparação entre os
estimadores amostrais e os de MV para a amostra normal aleatória e a
série de retornos padronizados do IBOVESPA. Para a amostra aleatória, a
curtose via MV é praticamente igual a 3 enquanto o estimador amostral é
menor que 3. A assimetria via MV é próxima de zero e o estimador
amostral tem duas casas decimais. A avaliação com a amostra aleatória é
interessante porque os parâmetros, de assimetria e curtos, são
conhecidos a priori, e os estimadores de MV são mais próximos dos
valores populacionais que os estimadores amostrais. Os resultados para a
série do IBOVESPA são mais interessantes porque neles a assimetria não
apresenta uma grande divergência, entretanto, as curtoses são bastente
diferentes. Contudo, a curtose amostral está dentro da região de
validade do estimador, logo, a FDP gerada com estes parâmetros é válida.

Ambos estimadores, amostrais e MV, apresentam valores válidos, contudo,
é importante avaliar distribuição produzida por estes estimadores. A
Figura \ref{fig:gc-dist} apresenta as funções de densidade geradas com
os parâmetros amostrais e com os de MV. No gráfico referente a amostra
aleatória, que é normal, observa-se um pequeno desvio entre as
distribuições geradas pelos dois estimadores (vermelha é MV e azul é
amostral). Entretanto, para a série do IBOVESPA as distribuições geradas
apresentam comportamentos bastante distintos. A distribuição com os
estimadores amostrais (azul) é multimodal de uma forma que não
representa os dados observados. Portanto, mesmo apresentando valores
válidos de assimetria e curtose para que \(p(x)\) seja uma FDP, o
resultado não gera uma distribuição que condiz com o comportamento
observado da variável.

Outra avaliação importante é a evolução dos estimadores ao longo do
tempo. Na Figura \ref{fig:skewness-kurtosis-ma} observamos o
comportamento dos estimadores amostrais ao longo do tempo em uma janela
móvel. A Figura \ref{fig:mle-skewness-kurtosis-ma} reproduz os
resultados da Figura \ref{fig:skewness-kurtosis-ma} utilizando os
estimadores de MV. Observa-se a estabilidade dos estimadores de MV,
pois, eles não apresentam grandes variações causadas por novos entradas
de retornos extremos na amostra.

\begin{table}

\caption{\label{tab:unnamed-chunk-14}\label{tab:mv-results} Resultados do cálculo dos estimadores de assimetria e curtose: máxima-verossimilhança e amostral, para a amostra aleatória normal e para a série de retornos padronizados do IBOVESPA.}
\centering
\begin{tabular}[t]{lrrrr}
\toprule
\multicolumn{1}{c}{ } & \multicolumn{2}{c}{Amostra aleatória} & \multicolumn{2}{c}{IBOVESPA} \\
\cmidrule(l{2pt}r{2pt}){2-3} \cmidrule(l{2pt}r{2pt}){4-5}
  & MLE & Amostra & MLE & Amostra\\
\midrule
Assimetria & 0.0000274 & 0.091725 & -0.1230408 & -0.1353708\\
Curtose & 3.0000166 & 2.847665 & 4.3409819 & 6.9598507\\
\bottomrule
\end{tabular}
\end{table}

\begin{figure}
\centering
\includegraphics{estimadores-assimetria-curtose_files/figure-latex/unnamed-chunk-16-1.pdf}
\caption{\label{fig:gc-dist} Os gráficos das funções de densidade de
probabilidade da expansão de Gram-Charlier, em vermelho temos os
estimadores de MV e em azul os estimadores amostrais. O gráfico (i) traz
a FDP que representa da amostra aleatória e (ii) representa a série de
retornos padronizados do IBOVESPA. Note a distribuição multimodal para
os retornos do IBOVESPA com os parâmetros amostrais.}
\end{figure}

\begin{figure}
\centering
\includegraphics{estimadores-assimetria-curtose_files/figure-latex/unnamed-chunk-18-1.pdf}
\caption{\label{fig:mle-skewness-kurtosis-ma} O gráfico mostra, na
primeira coluna, os estimadores de máxima verossimilhança de assimetria
e curtose em janela móvel de 504 dias úteis para a série de retornos
padronizados do IBOVESPA. Na segunda coluna, temos os estimadores
amostrais. Observa-se claramente a mudança no comportamento dos
estimadores. Os estimadores de máxima-verossimilhança são mais estáveis,
principalmente a curtose.}
\end{figure}

\section{Conclusão}\label{conclusao}

Os estimadores de MV para assimetria e curtose usando a expansão de
Gram-Charlier são calculados para uma amostra aleatória e para a série
de retornos padronizados do IBOVESPA em um intervalo de 17 anos.
Observa-se que, tanto para a série completa, como para sub-intervalos de
2 anos, os estimadores de MV são bem comportados. A assimetria e a
curtose estimados via máxima verossimilhança produzem uma distribuição
coerente com os dados observados, diferente dos estimadores amostrais.
Os resultados dos estimadores de MV aplicados a uma janela móvel também
mostram a estabilidade destes estimadores. As séries em janela móvel não
apresentam mudanças de nível abruptas semelhantes às séries dos
estimadores amostrais. Esta estabilidade é relevantes pois permite
avaliar mudanças de regime mais fundamentais no comportamento das
séries. Por exemplo, na Figura \ref{fig:mle-skewness-kurtosis-ma}
observa-se que a assimetria dos retornos do IBOVESPA muda de sinal entre
2010 e 2013, deixando de ser negativa. Este comportamento também está
presente no estimador amostral, entretanto, com um caráter mais
errático, tornando difícil caracterizar a mudança de regime. Na mesma
figura também é possível observar a curtose, que apresenta um
comportamento difícil de caracterizar nos estimadores amostrais,
enquanto nos estimadores de MV observa-se um crescimento suave entre
2005 e 2015.

A expansão de Gram-Charlier também é aplicada a formulações fechadas
para o apreçamento de opções, e desta forma, assimetria e curtose são
parâmetros do modelo, e portanto, a robustez na estimação é fator
determinante na aplicação do modelo.

A estibilidade dos parâmetros estimados via máxima verossimilhança e os
resultados coerentes quando se aplica estes parâmetros a expansão de
Gram-Charlier confirmam que estes estimadores são mais indicados para
medir assimetria e curtose do que os estimadores amostrais.
Principalmente quando aplicados ativos financeiros, como retorno de
ações. A utilização da expansão de Gram-Charlier é conveniente por ser
uma extensão da distribuição normal e por prover parâmetros claros para
avaliar a simetria da distribuição dos dados, assim como a presença de
eventos extremos, distantes da média.

\section*{Referências}\label{referencias}
\addcontentsline{toc}{section}{Referências}

\hypertarget{refs}{}
\hypertarget{ref-blinnikov1998}{}
{[}1{]} S. Blinnikov and R. Moessner, Astronomy and Astrophysics
Supplement Series \textbf{130}, 193 (1998).


\end{document}
